Im Gegensatz zu einem Resümee bzw. Fazit oder einem Review enthalten Inhaltsangaben keine Interpretationen und Bewertungen. Im Gegensatz zu Nacherzählungen dürfen Inhaltsangaben keine Spannungsbögen enthalten und werden in der Regel in der Gegenwart (Präsens, bei Vorzeitigkeit im Perfekt) abgefasst.\\

Da Inhaltsangaben in der Regel wesentlich kürzer als der Originaltext sein sollen, müssen sie zwangsläufig Teile des Inhalts auslassen. Sie können als Mittel der Sacherschließung dienen. Bei einem Buch, einer Dissertation oder Ähnlichem hat die Inhaltsangabe meist eine halbe bis eine Seite Umfang. Sie soll die wichtigsten Ergebnisse und verwendeten Methoden in allgemeiner (nicht zu spezieller) Fachsprache darstellen.