% !TeX encoding = UTF-8
\documentclass[a4paper,twoside,12pt,ngerman,openright,titlepage,halfparskip,headings=small,bibtotoc,pointlessnumbers,fleqn]{scrreprt} %[optionen]{Klasse}
%[..,zweiseitig,..,neue dt.Rechtschr.,chaptereröffnung nur auf ungeraden Seiten,mit Titelseite, halbe Leerzeile bei Absätzen,
% ..,kleine Überschriften,Literaturverzeichniseintrag ins Inhaltsverzeichnis,keine Punkte hinter chapternummern]

%%%%%%%%%%%%%%%%%%%%%%%%%%%%%%%%%%%%%%%%%%%%%%%%%%%%%%%%%%%%%%
% Packages
%%%%%%%%%%%%%%%%%%%%%%%%%%%%%%%%%%%%%%%%%%%%%%%%%%%%%%%%%%%%%%
\usepackage{emptypage}
\usepackage[utf8]{inputenc}
\usepackage[ngerman]{babel}										%neue deutsche Rechtschreibung
\usepackage[T1]{fontenc}
\usepackage[scaled]{uarial} 									%T1 Schriftsatz
\usepackage{hyperref}
\usepackage{graphicx}											%Einfügen von Bildern
\usepackage[left=3cm,right=3.0cm,top=1.25cm,bottom=1.25cm,includeheadfoot]{geometry}	%Seitenaufteilung
\usepackage{tabularx,array,booktabs,calc,multirow}				% Tabellen
\usepackage{mathcomp}
\usepackage{cite}
\usepackage{pdfpages}											% Einbindung von PDFs
\usepackage[babel,german=quotes]{csquotes}
\usepackage[font={normalsize,it}]{caption}
\usepackage{subcaption}										% Bei Bildern nebeneinander einzelne Bildunterschriften
\usepackage{xcolor}
\usepackage{textcomp}
\usepackage{paralist}											% Erweiterte Aufzählungsmöglichkeiten
\usepackage{xifthen}											% ifempty

%%%%%%%%%%%%%%%%%%%%%%%%%%%%%%%%%%%%%%%%%%%%%%%%%%%%%%%%%%%%%%
% Mathe-Pakete
%%%%%%%%%%%%%%%%%%%%%%%%%%%%%%%%%%%%%%%%%%%%%%%%%%%%%%%%%%%%%%
\usepackage{amsfonts}
\usepackage{amsmath}
\usepackage{amssymb}											% Symbole für Zahlenmengen usw.
\usepackage{cancel}												% Terme durchstreichen


%%%%%%%%%%%%%%%%%%%%%%%%%%%%%%%%%%%%%%%%%%%%%%%%%%%%%%%%%%%%%%
% Schriftart
%%%%%%%%%%%%%%%%%%%%%%%%%%%%%%%%%%%%%%%%%%%%%%%%%%%%%%%%%%%%%%
\usepackage{txfonts} 											% Schrift Times
 
%%%%%%%%%%%%%%%%%%%%%%%%%%%%%%%%%%%%%%%%%%%%%%%%%%%%%%%%%%%%%%
% Koma-Style
%%%%%%%%%%%%%%%%%%%%%%%%%%%%%%%%%%%%%%%%%%%%%%%%%%%%%%%%%%%%%%
\usepackage{scrpage2}
	% Kopfzeile
	\pagestyle{scrheadings} 									% eigener Seitenstil
	\clearscrheadings											% alle Voreinstellungen löschen
	\ohead{\pagemark}											% Seitenzahl oben außen
	\automark[chapter]{chapter} 								% \headmark definieren als Kapitelname
	\renewcommand{\chapterpagestyle}{scrheadings}				% Kapitelbeginn wie der Rest
	\rehead{\chaptername~\thechapter}							% Kapitelnummer   	
	\renewcommand{\chaptermark}[1]{\markright{#1}}	
	\lohead{\headmark}											% Kapitelname links auf ungeraden Seiten
	\lefoot{}\refoot{}\lofoot{}\rofoot{}\cehead{}\cohead{}		% keine Seitenzahlen in Fußzeilen
	\setheadsepline{0.2pt}										% Trennlinie zwischen Kopf und Textkörper
	\addtokomafont{pagehead}{\footnotesize}						% Schrift in Kopfzeile um einen Faktor kleiner als Standard
	\addtokomafont{pagehead}{\upshape}							% Schrift in Kopfzeile aufrecht (nicht kursiv)
	\addtokomafont{footnote}{\sffamily}							% Fußnoten serifenlos

%%%%%%%%%%%%%%%%%%%%%%%%%%%%%%%%%%%%%%%%%%%%%%%%%%%%%%%%%%%%%%
% Gestaltung der Überschriften
%%%%%%%%%%%%%%%%%%%%%%%%%%%%%%%%%%%%%%%%%%%%%%%%%%%%%%%%%%%%%%
	\parindent0cm																							
	\parskip\medskipamount	
	\renewcommand*{\chapterformat}{%
		\makebox[1.5cm][l]{\chapappifchapterprefix{\ }\thechapter\autodot\enskip}} 	% Chapter exakt 1.5cm hängend
	\renewcommand*{\othersectionlevelsformat}[1]{%
		\makebox[1.5cm][l]{\csname the#1\endcsname\autodot\enskip}}					% Section und Subsection hängend
	
	\RedeclareSectionCommand[
	beforeskip=-24pt,																%Abstand vor Chapter
	afterskip=12pt,																	%Abstand nach Chapter
	font=\bfseries\fontsize{14}{1.5\baselineskip}\selectfont]{chapter}				%Schriftgröße und Zeilenabstand
	
	\RedeclareSectionCommand[
	beforeskip=-24pt,
	afterskip=8pt,
	font=\bfseries\fontsize{13}{16}\selectfont]{section}
	
	\RedeclareSectionCommand[
	beforeskip=-24pt,
	afterskip=8pt,
	font=\bfseries\fontsize{12}{16}\selectfont ]{subsection}

%%%%%%%%%%%%%%%%%%%%%%%%%%%%%%%%%%%%%%%%%%%%%%%%%%%%%%%%%%%%%%
% Zahlen und Einheiten SIUNITX
%%%%%%%%%%%%%%%%%%%%%%%%%%%%%%%%%%%%%%%%%%%%%%%%%%%%%%%%%%%%%%
\usepackage{siunitx}								% Zahlen + Einheiten Bsp:\SI{9.81}{\meter\per\square\second}
	\sisetup{output-decimal-marker = {,}}
	\DeclareSIUnit[number-unit-product = \,] 
	\U{U}
	\DeclareSIUnit[number-unit-product = \,] 
	\Umin{\U \per \minute}
	\DeclareSIUnit[number-unit-product = \,] 
	\Us{\U \per \s}
	\DeclareSIUnit[number-unit-product = \,] 
	\Uss{\U \per \square\s}
	\DeclareSIUnit[number-unit-product = \,] 
	\decade{dek}

%%%%%%%%%%%%%%%%%%%%%%%%%%%%%%%%%%%%%%%%%%%%%%%%%%%%%%%%%%%%%%
% Befehl \vec für Vektorunterstriche
%%%%%%%%%%%%%%%%%%%%%%%%%%%%%%%%%%%%%%%%%%%%%%%%%%%%%%%%%%%%%%
\usepackage{accents}											% für Vektorunterstriche
	\renewcommand{\vec}[1]{\underaccent{\bar}{#1}}

%%%%%%%%%%%%%%%%%%%%%%%%%%%%%%%%%%%%%%%%%%%%%%%%%%%%%%%%%%%%%%
% Quellcode-Listing Einstellungen
%%%%%%%%%%%%%%%%%%%%%%%%%%%%%%%%%%%%%%%%%%%%%%%%%%%%%%%%%%%%%%
\usepackage{color,listings}
\usepackage{matlab-prettifier}
	\lstset{numbers=left, numberstyle=\tiny,language=[Sharp]C,showspaces=false, showstringspaces=false, breaklines=true,captionpos=b, tabsize=2}
	\lstset{literate=%
		{Ö}{{\"O}}1
		{Ä}{{\"A}}1
		{Ü}{{\"U}}1
		{ß}{{\ss}}1
		{ü}{{\"u}}1
		{ä}{{\"a}}1
		{ö}{{\"o}}1
		{~}{{\textasciitilde}}1
	}
	
	\definecolor{latexkeyword}{RGB}{0,149,255}
	\definecolor{codegray}{rgb}{0.5,0.5,0.5}
	
	\lstdefinestyle{myLatexStyle}{
		language=[LaTeX]TeX,
		frame=single,
		backgroundcolor=\color{white},
		rulecolor=\color{lightgray!40},
		breaklines=true,
		%  xleftmargin=\parindent,
		basicstyle=\footnotesize\ttfamily,
		keywordstyle=\bfseries\color{purple!40!black},
		commentstyle=\itshape\color{green!40!black},
		%identifierstyle=\color{blue},
		stringstyle=\color{orange}}

%%%%%%%%%%%%%%%%%%%%%%%%%%%%%%%%%%%%%%%%%%%%%%%%%%%%%%%%%%%%%%
% Eigene Environments
%%%%%%%%%%%%%%%%%%%%%%%%%%%%%%%%%%%%%%%%%%%%%%%%%%%%%%%%%%%%%%
\usepackage{environ}											% Neue Environments siehe Settings
	\NewEnviron{es}
	{
		\begin{equation}
		\begin{split}
		\BODY
		\end{split}
		\end{equation}
	}
	
	\NewEnviron{es*}
	{
		\begin{equation*}
			\begin{split}
				\BODY
			\end{split}
		\end{equation*}
	}
%%%%%%%%%%%%%%%%%%%%%%%%%%%%%%%%%%%%%%%%%%%%%%%%%%%%%%%%%%%%%%
% Mathtools Settings
%%%%%%%%%%%%%%%%%%%%%%%%%%%%%%%%%%%%%%%%%%%%%%%%%%%%%%%%%%%%%%
\usepackage{mathtools}											% Nummerierung von referenzierten Gleichungen
	\mathtoolsset{showonlyrefs} %Nur Gleichungen die mit \eqref{label} referenziert werden nummerieren (mathtools)


%%%%%%%%%%%%%%%%%%%%%%%%%%%%%%%%%%%%%%%%%%%%%%%%%%%%%%%%%%%%%%
% Nomenklatur
%%%%%%%%%%%%%%%%%%%%%%%%%%%%%%%%%%%%%%%%%%%%%%%%%%%%%%%%%%%%%%
\usepackage[acronym,nonumberlist,nomain]{glossaries} %Paket glossaries
\usepackage{glossary-super} %besonderer style, den ich gut finde 

\setlength{\glsdescwidth}{15cm}

\newglossary[slg]{symbolslist}{syi}{syg}{Symbolverzeichnis} % create add. symbolslist


\glsaddkey{unit}{\glsentrytext{\glslabel}}{\glsentryunit}{\GLsentryunit}{\glsunit}{\Glsunit}{\GLSunit}
\makeglossaries                                   % activate glossaries-package

\glssetnoexpandfield{unit}
\glsdisablehyper
\newglossarystyle{symbunitlong}{%
	\setglossarystyle{long3col}% base this style on the list style
	\renewenvironment{theglossary}{% Change the table type --> 3 columns
		\begin{longtable}{lp{0.7\glsdescwidth}>{\centering\arraybackslash}p{2cm}}}%
		{\end{longtable}}%
	%
	\renewcommand*{\glossaryheader}{%  Change the table header
		\bfseries Name & \bfseries Beschreibung & \bfseries Einheit \\
		\hline
		\endhead}
	\renewcommand*{\glossentry}[2]{%  Change the displayed items
		\glstarget{##1}{\glossentryname{##1}} %
		& \glossentrydesc{##1}% Description
		& $\left[\glsunit{##1}\right]$  \tabularnewline
	}
}

%%%%%%%%%%%%%%%%%%%%%%%%%%%%%%%%%%%%%%%%%%%%%%%%%%%%%%%%%%%%%%
% Inhaltsverzeichnis
%%%%%%%%%%%%%%%%%%%%%%%%%%%%%%%%%%%%%%%%%%%%%%%%%%%%%%%%%%%%%%
\usepackage{tocstyle}
\newtocstyle[KOMAlike][leaders]{alldotted}{}
\usetocstyle{alldotted}
\addtotoclist{toa} %Table of Appendix
\newcounter{appendix}[chapter]
\renewcommand{\theappendix}{\thechapter.\arabic{appendix}}

%%%%%%%%%%%%%%%%%%%%%%%%%%%%%%%%%%%%%%%%%%%%%%%%%%%%%%%%%%%%%%
% Plots aus Matlab mit pgfplots
%%%%%%%%%%%%%%%%%%%%%%%%%%%%%%%%%%%%%%%%%%%%%%%%%%%%%%%%%%%%%%
\usepackage{pgfplots}
\pgfplotsset{every axis/.append style={		
		ticklabel style={/pgf/number format/.cd, use comma, 1000 sep = {},fixed,precision=2}		
	}
}
\pgfplotsset{%
	every axis plot post/.append style={line width = 2pt}} 
%\pgfplotsset{execute at begin axis={\pgfplotsset{width=0.8\textwidth}}}

\usepackage{adjustbox}
\newcommand{\inserttikz}[6]{				% Befehl zum Einfügen von TIKZ
	\newlength{\figureheight} 
	\newlength{\figurewidth}
	\setlength{\figureheight}{#4}
	\setlength{\figurewidth}{#5}
	{
	\begin{figure}[{#6}]
		\centering 
			\begin{adjustbox}{max width=\textwidth}
				\input{#1}
			\end{adjustbox}
		\caption{{#3}}
		\label{#2}
	\end{figure}
	}
	\let\figureheight\relax
	\let\figurewidth\relax
}
% grid style
\pgfplotsset{grid style={line width=.1pt, draw=gray!10}}
\pgfplotsset{major grid style={line width=.2pt,draw=gray!50}}

%%%%%%%%%%%%%%%%%%%%%%%%%%%%%%%%%%%%%%%%%%%%%%%%%%%%%%%%%%%%%%
% Sonstige Settings
%%%%%%%%%%%%%%%%%%%%%%%%%%%%%%%%%%%%%%%%%%%%%%%%%%%%%%%%%%%%%%
	\parindent0.0cm													% Einrückung bei Absatz auf 0 setzen
% Tabellen
	\newcolumntype{L}[1]{>{\raggedright}p{#1}}						 %Linksbündiger Text in Tabellenzellen

%%%%%%%%%%%%%%%%%%%%%%%%%%%%%%%%%%%%%%%%%%%%%%%%%%%%%%%%%%%%%%
%Name für Bildunterschriften/Tabellenüberschriften
%%%%%%%%%%%%%%%%%%%%%%%%%%%%%%%%%%%%%%%%%%%%%%%%%%%%%%%%%%%%%%

	\renewcaptionname{ngerman}{\figurename}{\itshape Bild}
	\renewcaptionname{ngerman}{\tablename}{\itshape Tabelle}

%%%%%%%%%%%%%%%%%%%%%%%%%%%%%%%%%%%%%%%%%%%%%%%%%%%%%%%%%%%%%%
%Format der Nummerierungen
%%%%%%%%%%%%%%%%%%%%%%%%%%%%%%%%%%%%%%%%%%%%%%%%%%%%%%%%%%%%%%
	\renewcommand{\thefigure}{\arabic{chapter}-\arabic{figure}}
	\renewcommand{\thetable}{\arabic{chapter}-\arabic{table}}
	\renewcommand{\theequation}{\arabic{chapter}-\arabic{equation}} 

%%%%%%%%%%%%%%%%%%%%%%%%%%%%%%%%%%%%%%%%%%%%%%%%%%%%%%%%%%%%%%
% Formatierung der Bildunterschrift im Blocksatz (nicht zentriert)
%%%%%%%%%%%%%%%%%%%%%%%%%%%%%%%%%%%%%%%%%%%%%%%%%%%%%%%%%%%%%%

	\captionsetup{singlelinecheck=false,justification=justified}

%%%%%%%%%%%%%%%%%%%%%%%%%%%%%%%%%%%%%%%%%%%%%%%%%%%%%%%%%%%%%%
% Literaturverzeichnis
%%%%%%%%%%%%%%%%%%%%%%%%%%%%%%%%%%%%%%%%%%%%%%%%%%%%%%%%%%%%%%

	\bibliographystyle{alphadin}

%%%%%%%%%%%%%%%%%%%%%%%%%%%%%%%%%%%%%%%%%%%%%%%%%%%%%%%%%%%%%%
% Sonstige nützliche Pakete
%%%%%%%%%%%%%%%%%%%%%%%%%%%%%%%%%%%%%%%%%%%%%%%%%%%%%%%%%%%%%%
%\usepackage[abs]{overpic}										% Beschriftungen in Bildern
%\usepackage{subfigure}
%\usepackage{ulem}																	
%\usepackage{esvect}

%%%%%%%%%%%%%%%%%%%%%%%%%%%%%%%%%%%%%%%%%%%%%%%%%%%%%%%%%%%%%%
% Eigene Makros
%%%%%%%%%%%%%%%%%%%%%%%%%%%%%%%%%%%%%%%%%%%%%%%%%%%%%%%%%%%%%%
\newcommand{\todo}[1]{\noindent \textcolor{red}{\textbf{TODO: #1}}} % todo Markierung
\newcommand{\fe}[1]{\texttt{{\fontdimen2\font=0.25em\fontdimen4\font=0em}#1}} %definiert figure element um Begriffe aus einem Bild hervorheben zu können
\newcommand{\norm}[1]{\left\lVert#1\right\rVert}	%||x||
\newcommand{\chapterohnepagebreak}[1]{{\let\clearpage\relax\chapter{#1}}}
\newcommand{\chapteranhang}[1]{\chapter{#1}\addcontentsline{toa}{chapter}{\protect\numberline{\thechapter}#1}}
\newcommand{\sectionanhang}[1]{\section{#1}\addcontentsline{toa}{section}{\protect\numberline{\thesection}#1}}
\newcommand{\datenblatt}[5]{
	\sectionanhang{#1}
	\begin{figure}[h]
		\centering
		\begin{tabular}{@{}c@{\hspace{.5cm}}c@{}}
			\includegraphics[page=#3,width=.81\textwidth]{Bilder/anhang/#2}	
		\end{tabular}
		\caption{#4}
		\label{#5}
	\end{figure}
	\clearpage}

\newcommand{\messung}[4]{
	\sectionanhang{#1}	
	\inserttikz{Bilder/#2}{#4}{#3}{0.4\textwidth}{0.7\textwidth}{htbp}
	%\clearpage
}

